\documentclass[11pt, a4paper, oneside]{article}
\usepackage[margin=3cm]{geometry}
\usepackage{amsmath}
\usepackage{amsfonts}
\usepackage{MnSymbol}
\usepackage{appendix}
\usepackage{graphicx}
\usepackage{fancyhdr}
\usepackage{verbatim}
\usepackage[hidelinks]{hyperref}

\title{NSW Stage 6 Syllabus Critique}
\author{R. Mok
}
\date{\today}
\pagestyle{fancy}
\fancyhead[L]{NSW Stage 6 Syllabus}
\fancyfoot[C]{\thepage}

\newcommand{\lines}{\hbox to \linewidth{\leaders\hbox to 4pt{\hss . \hss}\hfil}\vspace{0.2cm}}
\begin{document}
\maketitle

\section{Motivation}
The NSW Mathematics Syllabus should achieve these following goals:
\begin{enumerate}
  \item Direct students in their development of correct Mathematical knowledge and applicable skills to solve real world problems;
  \item Prepare students who wish to pursue further study in Science, Technology, Mathematics and Engineering (STEM) fields in university;
  \item Foster a positive culture of Mathematics education in NSW.
\end{enumerate}
At the moment, the Syllabus as it stands does not achieve these goals. Teachers currently battle with the Syllabus in trying to teach all of the content in time for the HSC Examinations while balancing rigour of treatment, designing Assessments for and of learning and of course day to day school life and administration. To put it succinctly, the feedback from many teachers is that there is too much to cover. With limited time and a large amount of material to cover, the quality of a student's learning experience of Mathematics ultimately suffers. As it stands, too much content is a sign that a clearer direction is needed - what do we want students to know when they complete the HSC course?

At the moment, the answer to that question is: a bit of everything - so much so that the course can be described as a ``jack of all trades, master of none''. We need a better strategy than to just throw everything in!

Furthermore, some treatments of Mathematical content in the Syllabus are incorrect, incomplete, vague or outdated. How can students feel prepared for university when one of their first lectures on Functions is one informing them that their prior learning of the topic is wrong? Does it not destroy students' trust in the integrity of their secondary education when our tertiary cousins contradict or greatly supersede their prior learning?

In the sections to follow, I have set out the current syllabus of the Preliminary and HSC courses to one list to give a brief overview of what we expect students to know (depending on whether they take Advanced, Extension I and/or Extension II). Then, I make some criticisms on the current Syllabus and recommendations for future reformations.

\section{Current Framework}
The content outcomes of the Syllabus can be structured into the following outline. Key: (A): Advanced, (X1): Extension 1, (X2): Extension 2. Note that in no way does brevity of a dot-point below imply a short time frame in the teaching of the concept.
\begin{enumerate}
  \item (A) Algebra Over $\mathbb{R}$: manipulating indices, surds, algebraic fractions, quadratic expressions, solving equations;
  \item Real Valued Relations and Functions (i.e. subsets of the number plane $\mathbb{R}^2$):
    \begin{enumerate}
      \item (A) Definitions of Relation and Function
      \item (A, X1) Graphs and their transformations
      \item (A) Linear, Quadratic, Cubic, Polynomial, Absolute Value, Rational, Circle, Semi-circle Functions/Relations
      \item (X1) Inverse Functions
      \item (X1) Parametric Curves $\gamma: I \rightarrow \mathbb{R}^2$
      \item (X1) Algebra of Polynomial Expressions and Roots of Polynomial Equations
      \item (A, X1) Trigonometry
      \item (A) Exponential and Logarithmic Functions
    \end{enumerate}
  \item Differential Calculus of Functions in One Variable
    \begin{enumerate}
      \item (A) Motivation and `Definitions'
      \item (A) Product, Quotient, Chain Rule
      \item Applications
        \begin{enumerate}
          \item (A) Graphs of Functions
          \item (X1) Rates of Change
        \end{enumerate}
    \end{enumerate}
  \item Integral Calculus
    \begin{enumerate}
      \item (A) Motivation and Definitions
      \item (X1, X2) Further Skills and Techniques
      \item Applications
        \begin{enumerate}
          \item (X1) Volumes of Revolution
          \item (X1) Solving differential equations
          \item (X2) Mechanics
        \end{enumerate}
    \end{enumerate}
  \item Statistical Analysis
    \begin{enumerate}
      \item (A) Set Notation and Venn Diagrams
      \item (A) Probabilities of Sample Spaces
      \item (A) Univariate data analysis
      \item (A) Bivariate data analysis
      \item (A) Discrete Random Variables
        \begin{enumerate}
          \item (X1) Binomial Distribution
        \end{enumerate}
      \item (A) Continuous Random Variables
        \begin{enumerate}
          \item (A) Normal Distribution
        \end{enumerate}
    \end{enumerate}
  \item Sequences and Series
    \begin{enumerate}
      \item Arithmetic Series and Sequences
      \item Geometric Series and Sequences
      \item Applications to Financial Modelling
    \end{enumerate}
  \item (X1) Combinatorics: permutations, combinations, pigeonhole principle, binomial theorem
  \item Proof
    \begin{enumerate}
      \item (X2) The Nature of Proof: informal treatment of first order logic, rules of deduction, inequalities.
      \item (X1, X2) Proof by Mathematical Induction
    \end{enumerate}
  \item (X1, X2) (Euclidean) Vectors
  \item (X2) Complex Numbers
\end{enumerate}

\section{Criticisms and Recommendations}
\begin{itemize}
  \item The rigorous development of Logic and Set Theory as the Foundation of Mathematics was a major theme of the early 20th century, and any level of Mathematics in the modern zeitgeist cannot avoid it. The importance of the work of the Mathematicians in that era has resulted in a language, of Logic and Sets, that describes mathematical concepts with a degree of precision that those who came before them did not have - not even Isaac Newton, Leonard Euler or Evariste Galois had access to such language in their day and doing Mathematics in that time was highly inefficient compared to now.

    The almost complete absence of Set Theory in our Syllabus until the Statistical Analysis section, and saving a vague introduction to first order logic only for Mathematics Extension 2's topic of Proof is a symptom of being stuck in an old era of thinking before the early 20th century. We currently live more than 20 years into the 21st century, when are we going to update our Syllabus to match the modern treatment of the Foundations of Mathematics to include the language of Sets and Logic?

    For a reformed Syllabus, Set Theory and Logic should be forefront and centre before any other work is completed. Of course, the question begs to be asked, ``are students taking Mathematics Advanced capable of learning this language?''. I believe the answer is yes! The level of basic set algebra, i.e. unions, intersections, deletions, cartesian products, subsets, set equality and set membership is arguably more manageable than some complex algebraic manipulations that they are currently required to do in the Syllabus.

    With regard to the language of logic, the language of proof dot-point in the Extension 2 Syllabus can be further broken down into parts across the Mathematics Advanced, Extension 1 and Extension 2 courses commensurate with the difficulty of the course. For example, students in Mathematics Advanced can focus on understanding `or' ($\lor$), `and' ($\land$), negation ($\neg$), implication ($\Rightarrow$), equivalence ($\Leftrightarrow$), `for all' ($\forall$) and `there exists' ($\exists$). Students in Extension 1 can further develop their understanding of propositions with converse, contrapositive, negations, some Rules of Replacement and Truth Tables. Students in Extension 2 can further develop their understanding with Rules of Deduction, and manipulating more complex compound propositions and predicates. I understand that this perspective may be controversial right now, but at the very least the use of this language in the classroom should be regular in the pedagogy of Mathematics even if the decision is made to keep its explicit notational form, assessment and deeper study in the Extension 2 course.

    While this may seem like a lot of setup for something abstract or perhaps `too foundational' - the rewarding effects are profound in the context of STEM, including but not limited to the following applications:
    \begin{itemize}
      \item Computer Programmers must think with conditional statements, e.g.
        \begin{verbatim}
if x > 0:
  print(x)
else:
  print(2*x)
        \end{verbatim}
        Understanding logic is an important skill they require daily so to be taught how to do so in our courses would greatly benefit them.
      \item All other Mathematical content a student learns henceforth has a coherent and consistent framework/language from which it is built. The implication of this is that the NSW Syllabus can afford to take out some concepts to make it less bloated - leave these omissions or a student's further study in their own extra-curricular research or university course.
      \item Just like how learning to read English is the key that opens a world of books to the reader, the student who learns to read and think in Sets and Logic can pick up any Mathematics book not necessarily catered for the NSW curriculum and know how to decipher its contents to learn. This opens up a lot more learning opportunities and resources for the student (and teacher alike) to practice!
    \end{itemize}
  \item As it stands, the definitions of Relation and Function in the Syllabus is not updated to match the set theoretic treatments that universities today must reteach to first year students.
    \begin{enumerate}
    \item The concept of Relation should be updated to match the modern set theoretic treatment:
      \begin{center}
        A Relation $R$ over sets $A$ and $B$ is a subset of $A \times B$.
      \end{center}
    \item The concept of Function should be updated to match the modern set theoretic treatment:
      \begin{center}
        A Function $f: A \rightarrow B$ is a Relation $f$ over sets $A$ and $B$ and for all $a \in A$ there exists a unique $b \in B$ such that $(a,b) \in f$. $(a,b) \in f$ is equivalent to $f(a) = b$.

        $A$ is the domain, and $B$ is the co-domain of the function $f$. The range of a function (also known as image) is the set $f(A) = \{f(a) : a \in A\}$.
      \end{center}
    \item Students in the Extension 2 course could study Equivalence Relations and Equivalence Classes, which has important applications to many areas of Mathematics.
    \end{enumerate}
  \item With the set theoretic framework, a big bulk of Preliminary coursework is building up an understanding of different types of Relations over $\mathbb{R}^2$ and Functions $f: A \subseteq \mathbb{R} \rightarrow \mathbb{R}$. For example:
    \begin{enumerate}
      \item Quadratic Functions $f: \mathbb{R} \rightarrow \mathbb{R}: f(x) = ax^2 + bx + c$.
      \item Hyperbolic Functions $f: \mathbb{R}\backslash\{0\} \rightarrow \mathbb{R}: f(x) = \frac{1}{x}$.
      \item Circle Relations $\{ (x,y) \in \mathbb{R}^2: x^2 + y^2 = r^2 \}$.
    \end{enumerate}
  \item It should also be noted at the moment when HSC students write, for sake of example, the domain of a function $f(x) = x^2$ as ``$x \geq 0$'', this is actually an abuse of notation - what they really mean is $\{ x \in \mathbb{R}: x \geq 0\}$. This is made more clear when Sets are taught properly.
  \item Currently, students believe that a necessary and sufficient condition for a function to have an inverse is that $f$ is one-to-one - which is in general not true. This is because the current Syllabus defaults to setting all functions to be $f: A \subseteq \mathbb{R} \rightarrow f(A) \subseteq \mathbb{R}$ to avoid dealing with the co-domain concept. In the absence of the co-domain concept, an incorrect theorem such as this is being brought into universities that then need to be (sometimes painstakingly) undone\footnote{There's always that one student who asks `so the co-domain is just the range?' after an hour lecture on the subject.}. Shouldn't we make sure there is a smooth connection between exiting from high school mathematics and entering into tertiary mathematics?

    While we have the term `one-to-one' (injective) in the current Syllabus, we do not yet have the term `onto' (surjective) which means that the range of the function is equal to its co-domain, nor the term `one-to-one correspondence' (bijective) that describes a function that is both `one-to-one' and `onto'.

  \item The concept of Sequences and Series being placed under the umbrella topic title of Financial Mathematics is bewildering. It shows a complete misplacement of importance of the concept in the Syllabus. The application of Sequences and Series to Financial Modelling is but \emph{one} use of this concept that is ubiquitous in other areas of mathematical study: we don't need to look far to see its use in the computation of $Var(X) = \sum_{i=1}^n xP(X=x)$ when dealing with discrete random variables.

    Furthermore, real number sequences that can be represented as an ordered list: $(x_0, x_1, x_2, \ldots)$ are just functions $x: \mathbb{N} \rightarrow \mathbb{R}$. Again, we see that the set theoretical framework makes fast work of defining clearly what these objects are.

  \item The removal of the concept of Limits and calculations of limits from the old syllabus when the current syllabus was written is a step backwards. I would like to see this placed back into the course. Concepts such as continuity and differentiation rely on limits - how can this concept be blatantly replaced by `an informal treatment' that is seen only in the context of the difference quotient?

    Furthermore, the language of logic and sets would allow more able students (perhaps those who take the Extension 2 course) to eventually learn the formal definition of limits of sequences: $\lim_{n\rightarrow \infty} x_n$ and limits of real valued functions: $\lim_{x \rightarrow a} f(x) = c$ with the usual $\varepsilon$-$\delta$ definitions.

  \item The topics of Univariate and Bivariate Data Analysis take up a lot of time in the teaching program that could be spent developing a deeper understanding of the other concepts in the course. I recommend that this topic be left in Stage 5 because a proper Stage 6 course that actually iterates upon Stage 5 dot-points should include Hypothesis Testing (as opposed to only Pareto Charts as it is currently). However, given the amount of content already in the course, its inclusion would be most unwise.

  \item Vectors to Physicists and to Engineers are objects that have direction and length but \emph{not} to Mathematicians! These careers work in special sets of vectors called Hilbert Spaces such as $\mathbb{R}^2$ and $\mathbb{R}^3$, which is why they are defined that way by them.

    If we want to introduce the concept of Vectors \emph{correctly} to Mathematics students, we need the concept of sets of vectors i.e. Vector Spaces, whose elements can be added together to form another vector, and multiplied by a scalar to form another vector. For example, the set of polynomials with real coefficients, the set of functions with domain $[0,1]$, the set $\{ a + b \sqrt{2} : a,b \in \mathbb{Q} \}$, the set of complex numbers $\mathbb{C} = \{ a + ib : a, b \in \mathbb{R} \mbox{ and } i^2 = -1 \}$. These are some examples of vector spaces.

    The proper term for vectors that have direction and length in $\mathbb{R}^2$ and $\mathbb{R}^3$ (Euclidean) vector spaces is the `Euclidean Vector'. In here, one may define the concepts of dot product and vector norm (length).

  \item A set theoretic understanding of vector spaces in high school would improve the teaching and learning experience of students when they take a first year Linear Algebra course in University as they study transformations as linear functions from a vector space $V$ into another vector space $W$, i.e. $T: V \rightarrow W$. The composition of such functions can be represented by matrix algebra. To this note, I do not recommend introducing the concept of Matrices to high school students for the purpose of solving linear equations.

  \item The concept of a Random Variable in the Syllabus and glossary pages is quite vaguewithout precise set theoretic language - ``a random variable is a variable...'' is not a great start to a sentence. Again we see that a set theoretic framework makes it very easy to define a (real) Random Variable - it is a function $X: \Omega \rightarrow \mathbb{R}$ where $\Omega$ is the sample space of the probability experiment. $X$ is called a discrete random variable if and only if its range is \emph{countable} and $X$ is called a continuous random variable if and only if its range is \emph{uncountably infinite}. These concepts of countable and uncountably infinite can be unpacked here or in the definition of functions.

  \item In the current Syllabus, there is a glaring omission of the Normal Approximation to the Binomial Distribution, yet it then skips over to applying a Normal Approximation to the Sample Proportion. This is like running before learning to walk! What is really going on here is the Central Limit Theorem. I suppose it has been removed from even basic lip service because the concept of a limit is no longer covered in the course - all the more reason to bring it back in.

  \item As mentioned earlier, with the default functions of study in the HSC being of the form $f: A \subseteq \mathbb{R} \rightarrow f(A) \subseteq \mathbb{R}$, many students and teachers alike confuse Complex valued functions $f: A \subseteq \mathbb{C} \rightarrow \mathbb{C}$ to behave the same way as their real counterparts. For example $\exp: \mathbb{C} \rightarrow \mathbb{C}$ no longer behaves the same way as $\exp: \mathbb{R} \rightarrow \mathbb{R}$! Where the real version is one-to-one, the complex version is not - this is easy to check with $e^0 = 1$ and $e^{2\pi i} = 1$.

    A more set theoretic approach to defining functions can quickly explain the previously mysterious misbehaviour of these complex valued functions. Students may be referred to courses in Complex Analysis in universities and/or textbooks to discover more wondrous and weird properties of these functions.

  \item The topic of Combinatorics is heavily dependent on Sets. For example, counting the number of elements in a union of sets, such as $A \cup B$ can be broken down to sums of the size of their intersections (the inclusion-exclusion principle):
    \[ |A \cup B| = |A| + |B| - |A \cap B| \]
    \[ |A \cup B \cup C| = |A| + |B| + |C| - |A \cap B| - |A \cap C| - |B\cap C| + |A \cap B \cap C| \]
    etc.

    This concept appears in the topic of Probability at the moment but seeing it in the context of Sets is more general than restricted to the one topic. Using these tools is much more powerful than trying to come up with a logic by oneself - a great example of this at play is the question in the Cambridge Textbook about Bob's 8 shirts.

    Moreover, permutations of a finite set $X$ are one-to-one correspondences (bijective functions) of $X$ to itself, i.e. $\sigma: X \tilde{\rightarrow} X$. The method of counting these permutations, and equivalent permutations, is actually done quite well in the HSC course. It does a fairly good job at equipping students to solve problems like how many ways can $n$ objects be placed into $m$ buckets: the pigeonhole principle states that there does not exist any one-to-one (injective) function of the set of $n$ objects into the set of $m$ buckets, and counting how many ways $n$ people can be placed into $m$ rooms such that no room is empty is equivalent to counting the number of onto (surjective) functions.
\end{itemize}

\section{Concluding Remarks}

\end{document}
